\documentclass[11pt,a4paper]{article}
\usepackage[margin=0.9in]{geometry}
\usepackage{booktabs}
\usepackage{listings}
\usepackage{xcolor}
\usepackage{hyperref}

\lstset{basicstyle=\ttfamily\small, backgroundcolor=\color{gray!8},
  frame=single, breaklines=true, columns=fullflexible, keepspaces=true}

\title{Wednesday Demo: What's Ready}
\author{Noumenal Labs}
\date{February 11, 2026}

\begin{document}
\maketitle

\section{Quick Start}

\begin{lstlisting}[language=bash]
cd topological-blankets

# Dry-run (numpy + matplotlib only)
python ralph/experiments/wednesday_demo.py --dry-run

# Live (needs gymnasium-robotics + MuJoCo)
python ralph/experiments/wednesday_demo.py --live --planner symbolic

# Live with TB-discovery pipeline
python ralph/experiments/wednesday_demo.py --live --planner tb-discover

# With 3D trajectory inset (works with any mode)
python ralph/experiments/wednesday_demo.py --dry-run --show-3d
\end{lstlisting}

Outputs go to \texttt{ralph/results/}: timestamped GIF, JSON, and 8 key-frame PNGs.

\section{The Six Phases}

\begin{tabular}{clll}
\toprule
\textbf{\#} & \textbf{Phase} & \textbf{Steps} & \textbf{What Happens} \\
\midrule
1 & Autonomous Push     & 0--19  & Ensemble solves push, low uncertainty \\
2 & Novel Configuration & 20     & Object displaced to unfamiliar position \\
3 & Uncertainty Spike   & 21--31 & Ensemble disagrees, catastrophe fires \\
4 & Human Takeover      & 32--51 & Two goal injections (steps 32, 42) \\
5 & Collaborative       & 52--79 & Agent resumes, task completes \\
6 & Structure Analysis  & final  & TB coupling: pre vs.\ post perturbation \\
\bottomrule
\end{tabular}

\medskip\noindent
All 6/6 acceptance criteria pass. Planner options: \texttt{symbolic} (default),
\texttt{tb} (ground-truth partition), \texttt{tb-discover} (full pipeline).

\section{Integration Surface}

Single file: \texttt{ralph/experiments/wednesday\_demo.py} (1500 lines).

\subsection{Config}

\begin{lstlisting}[language=python]
@dataclass
class WednesdayDemoConfig:
    max_steps: int = 80
    env_id: str = "FetchPush-v4"
    perturbation_step: int = 20
    first_goal_step: int = 32
    second_goal_step: int = 42
    release_step: int = 52
    yellow_threshold: float = 0.35
    red_threshold: float = 0.60
    ghost_horizon: int = 15
    n_ghost_members: int = 5
    fig_width: float = 16.0
    fig_height: float = 9.0
    dpi: int = 100
    gif_fps: int = 2
    show_3d: bool = False
\end{lstlisting}

\subsection{Run and Save}

\begin{lstlisting}[language=python]
records, frames = run_dry_run_demo(cfg)
# or
records, frames = run_live_demo(cfg, planner_mode="symbolic")

# records: list[StepRecord] -- per-step state
#   .grip_pos, .obj_pos, .desired_goal   (3D arrays)
#   .obs_vec                             (25D)
#   .ensemble_disagreement, .cem_plan_spread
#   .catastrophe_severity, .catastrophe_level
#   .phase_name, .teleop_mode, .human_goal
#   .is_key_frame, .key_frame_label
#
# frames: list[np.ndarray] -- (h, w, 3) uint8

summary = save_demo_outputs(records, frames, cfg)
# Writes GIF + PNGs + JSON to ralph/results/
\end{lstlisting}

\subsection{Frame Rendering}

Two functions return raw RGB numpy arrays:

\begin{lstlisting}[language=python]
def render_demo_frame(step_record, history, tb_grouping,
                      config, phase, ghost_positions=None,
                      ghost_3d=None) -> np.ndarray:
    """(h, w, 3) uint8 frame for phases 1-5.
    Replace this to plug in a custom renderer."""

def render_structure_frame(coupling_before, coupling_after,
                           config, step) -> np.ndarray:
    """(h, w, 3) uint8 frame for phase 6."""
\end{lstlisting}

\subsection{Ghost Trajectories}

\begin{lstlisting}[language=python]
def generate_synthetic_ghosts(grip_pos, obj_pos, phase,
    n_members=5, horizon=15, rng=None) -> np.ndarray:
    """(n_members, horizon, 2) XY. Spread: 0.08 spike, 0.01 normal."""

def generate_synthetic_ghosts_3d(grip_pos, obj_pos, phase,
    n_members=5, horizon=15, rng=None) -> np.ndarray:
    """(n_members, horizon, 3) XYZ. Same spread logic."""

def generate_synthetic_coupling(phase, n_vars=25, seed=42
    ) -> np.ndarray:
    """(25, 25) symmetric. phase='before': block-diagonal,
    phase='after': diffuse (perturbation breaks structure)."""
\end{lstlisting}

\section{Available Components}

Importable if \texttt{panda} and \texttt{topological\_blankets} are on
\texttt{sys.path}:

\begin{lstlisting}[language=python]
# Catastrophe detection
from panda.catastrophe_bridge import (
    CatastropheBridge, CatastropheSignal, HandoverState)

# Planners
from panda.symbolic_planner import make_symbolic_planner
from panda.learned_planner import (
    TBGuidedPlanner, make_fetchpush_ground_truth_partition)

# TB discovery from ensemble Jacobians
from panda.tb_discovery import discover_or_fallback

# Core TB library (v0.2.0)
from topological_blankets import TopologicalBlankets, compute_eigengap
\end{lstlisting}

\section{BayesformerTB: Negative Result (Not a Blocker)}

TB-masked attention transformer on the GHMM task (5-factor, 433-vocab,
10k sequences, 40 epochs). Architecture is implemented correctly; the
performance hypothesis is falsified.

\begin{tabular}{lcc}
\toprule
& \textbf{BayesformerTB} & \textbf{Vanilla GPT-2} \\
\midrule
Final CE loss        & 4.56   & 1.77 \\
Training time        & 914\,s & 190\,s \\
Predictive entropy   & 5.06   & 2.15 \\
\bottomrule
\end{tabular}

\medskip\noindent
TB detects factored structure post-hoc (eigengap works), but imposing it as an
architectural constraint during training hurts performance. The soft attention
mask plus eigengap regularization ($\lambda=0.01$) is too restrictive.

The \texttt{SmallGPT2} model (4 layers, $d=120$, 4 heads, 750k params) and
\texttt{GHMMDataset} from \texttt{fwh\_ghmm\_tb\_detection.py} are reusable.

\section{File Map}

\begin{lstlisting}
ralph/
  experiments/
    wednesday_demo.py          # main demo
    bayesformer_tb.py          # BayesformerTB experiment
    fwh_ghmm_tb_detection.py   # GHMM + TB detection
    demo_push_visualization.py # shared viz utilities
  results/                     # all outputs land here
  WEDNESDAY_DEMO_GUIDE.md      # detailed guide + troubleshooting
  FWH_TB_DETECTION.md          # TB <-> FWH bridge document
  prd.json                     # full PRD (114 user stories)

topological_blankets/          # core TB library v0.2.0
  core.py, spectral.py, detection.py, clustering.py, ...
\end{lstlisting}

\end{document}
